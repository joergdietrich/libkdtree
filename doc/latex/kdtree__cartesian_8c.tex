\hypertarget{kdtree__cartesian_8c}{
\subsection{lib/kdtree\_\-cartesian.c File Reference}
\label{kdtree__cartesian_8c}\index{lib/kdtree\_\-cartesian.c@{lib/kdtree\_\-cartesian.c}}
}
Routines specific to the n-dimensional cartesian kd-tree. 

\subsubsection*{Functions}
\begin{CompactItemize}
\item 
struct \hyperlink{structkdNode}{kdNode} $\ast$ \hyperlink{kdtree__cartesian_8c_87b949986bab7fbea42e703eb070f8eb}{kd\_\-buildTree} (struct kd\_\-point $\ast$points, unsigned long nPoints, void $\ast$($\ast$constr)(void $\ast$), void($\ast$destr)(void $\ast$), float $\ast$min, float $\ast$max, int dim, int max\_\-threads)
\begin{CompactList}\small\item\em build kd-tree structure \item\end{CompactList}\item 
struct \hyperlink{structpqueue}{pqueue} $\ast$ \hyperlink{kdtree__cartesian_8c_617ec34cd57f74373f2b076f68bb0725}{kd\_\-ortRangeSearch} (struct \hyperlink{structkdNode}{kdNode} $\ast$node, float $\ast$min, float $\ast$max, int dim)
\begin{CompactList}\small\item\em Perform orthogonal range search (get all points in a hyperrectangle). \item\end{CompactList}\item 
struct \hyperlink{structkdNode}{kdNode} $\ast$ \hyperlink{kdtree__cartesian_8c_c1a9dd9e9287a3527a69aed20c1661d3}{kd\_\-nearest} (struct \hyperlink{structkdNode}{kdNode} $\ast$node, float $\ast$p, float $\ast$max\_\-dist\_\-sq, int dim)
\begin{CompactList}\small\item\em Find the nearest neighbor of a point. \item\end{CompactList}\item 
struct \hyperlink{structpqueue}{pqueue} $\ast$ \hyperlink{kdtree__cartesian_8c_b974c9c2fb71896b9b044f340f6afc56}{kd\_\-qnearest} (struct \hyperlink{structkdNode}{kdNode} $\ast$node, float $\ast$p, float $\ast$max\_\-dist\_\-sq, unsigned int q, int dim)
\begin{CompactList}\small\item\em Return the q nearest-neighbors to a point. \item\end{CompactList}\item 
struct \hyperlink{structpqueue}{pqueue} $\ast$ \hyperlink{kdtree__cartesian_8c_23e1d72de397f86f0b13b2cc4d6c72d0}{kd\_\-range} (struct \hyperlink{structkdNode}{kdNode} $\ast$node, float $\ast$p, float $\ast$max\_\-dist\_\-sq, int dim, int ordered)
\begin{CompactList}\small\item\em Perform a range search around a point. \item\end{CompactList}\end{CompactItemize}


\subsubsection{Detailed Description}
Routines specific to the n-dimensional cartesian kd-tree. 



\subsubsection{Function Documentation}
\hypertarget{kdtree__cartesian_8c_87b949986bab7fbea42e703eb070f8eb}{
\index{kdtree\_\-cartesian.c@{kdtree\_\-cartesian.c}!kd\_\-buildTree@{kd\_\-buildTree}}
\index{kd\_\-buildTree@{kd\_\-buildTree}!kdtree_cartesian.c@{kdtree\_\-cartesian.c}}
\paragraph[kd\_\-buildTree]{\setlength{\rightskip}{0pt plus 5cm}struct {\bf kdNode}$\ast$ kd\_\-buildTree (struct kd\_\-point $\ast$ {\em points}, \/  unsigned long {\em nPoints}, \/  void $\ast$($\ast$)(void $\ast$) {\em constr}, \/  void($\ast$)(void $\ast$) {\em destr}, \/  float $\ast$ {\em min}, \/  float $\ast$ {\em max}, \/  int {\em dim}, \/  int {\em max\_\-threads})\hspace{0.3cm}{\tt  \mbox{[}read\mbox{]}}}\hfill}
\label{kdtree__cartesian_8c_87b949986bab7fbea42e703eb070f8eb}


build kd-tree structure 

\begin{Desc}
\item[Parameters:]
\begin{description}
\item[{\em points}]an array of kd\_\-points (struct with position vector and data container).\item[{\em nPoints}]the length of the points array.\item[{\em constr}]a pointer to a void $\ast$constructor() function to include the data container in the tree; optional, can be NULL\item[{\em destr}]a pointer to a void destructor() function to free() the data containers in the tree; optional, can be NULL, but should be given if the constr argument is non-NULL.\item[{\em min}]a vector with the minimum positions of the corners of the hyperrectangle containing the data.\item[{\em max}]a vector with the maximum positions of the corners of the hyperrectangle containing the data.\item[{\em dim}]the dimensionality of the data.\item[{\em max\_\-threads}]the maximal number of threads spawned for construction of the tree. The threads will be unbalanced if this is not a power of 2.\end{description}
\end{Desc}
\begin{Desc}
\item[Returns:]root node of the tree \end{Desc}
\hypertarget{kdtree__cartesian_8c_c1a9dd9e9287a3527a69aed20c1661d3}{
\index{kdtree\_\-cartesian.c@{kdtree\_\-cartesian.c}!kd\_\-nearest@{kd\_\-nearest}}
\index{kd\_\-nearest@{kd\_\-nearest}!kdtree_cartesian.c@{kdtree\_\-cartesian.c}}
\paragraph[kd\_\-nearest]{\setlength{\rightskip}{0pt plus 5cm}struct {\bf kdNode}$\ast$ kd\_\-nearest (struct {\bf kdNode} $\ast$ {\em node}, \/  float $\ast$ {\em p}, \/  float $\ast$ {\em max\_\-dist\_\-sq}, \/  int {\em dim})\hspace{0.3cm}{\tt  \mbox{[}read\mbox{]}}}\hfill}
\label{kdtree__cartesian_8c_c1a9dd9e9287a3527a69aed20c1661d3}


Find the nearest neighbor of a point. 

\begin{Desc}
\item[Parameters:]
\begin{description}
\item[{\em node}]the root node of the tree to be searched.\item[{\em p}]a vector to the point whose nearest neighbor is sought.\item[{\em max\_\-dist\_\-sq}]the square of the maximum distance to the nearest neighbor.\item[{\em dim}]the dimension of the data.\end{description}
\end{Desc}
\begin{Desc}
\item[Returns:]A pointer to node containing the nearest neighbor. max\_\-dist\_\-sq is set to the square of the distance to the nearest neigbor. \end{Desc}
\hypertarget{kdtree__cartesian_8c_617ec34cd57f74373f2b076f68bb0725}{
\index{kdtree\_\-cartesian.c@{kdtree\_\-cartesian.c}!kd\_\-ortRangeSearch@{kd\_\-ortRangeSearch}}
\index{kd\_\-ortRangeSearch@{kd\_\-ortRangeSearch}!kdtree_cartesian.c@{kdtree\_\-cartesian.c}}
\paragraph[kd\_\-ortRangeSearch]{\setlength{\rightskip}{0pt plus 5cm}struct {\bf pqueue}$\ast$ kd\_\-ortRangeSearch (struct {\bf kdNode} $\ast$ {\em node}, \/  float $\ast$ {\em min}, \/  float $\ast$ {\em max}, \/  int {\em dim})\hspace{0.3cm}{\tt  \mbox{[}read\mbox{]}}}\hfill}
\label{kdtree__cartesian_8c_617ec34cd57f74373f2b076f68bb0725}


Perform orthogonal range search (get all points in a hyperrectangle). 

\begin{Desc}
\item[Parameters:]
\begin{description}
\item[{\em node}]the root node of tree to be searched.\item[{\em min}]a vector with the minimum positions of the corners of the hyperrectangle containing the data.\item[{\em max}]a vector with the maximum positions of the corners of the hyperrectangle containing the data.\item[{\em dim}]the dimension of the data.\end{description}
\end{Desc}
\begin{Desc}
\item[Returns:]Pointer to a priority queue, NULL in case of problems. \end{Desc}
\hypertarget{kdtree__cartesian_8c_b974c9c2fb71896b9b044f340f6afc56}{
\index{kdtree\_\-cartesian.c@{kdtree\_\-cartesian.c}!kd\_\-qnearest@{kd\_\-qnearest}}
\index{kd\_\-qnearest@{kd\_\-qnearest}!kdtree_cartesian.c@{kdtree\_\-cartesian.c}}
\paragraph[kd\_\-qnearest]{\setlength{\rightskip}{0pt plus 5cm}struct {\bf pqueue}$\ast$ kd\_\-qnearest (struct {\bf kdNode} $\ast$ {\em node}, \/  float $\ast$ {\em p}, \/  float $\ast$ {\em max\_\-dist\_\-sq}, \/  unsigned int {\em q}, \/  int {\em dim})\hspace{0.3cm}{\tt  \mbox{[}read\mbox{]}}}\hfill}
\label{kdtree__cartesian_8c_b974c9c2fb71896b9b044f340f6afc56}


Return the q nearest-neighbors to a point. 

\begin{Desc}
\item[Parameters:]
\begin{description}
\item[{\em node}]the root node of the tree to be searched.\item[{\em p}]a vector to the point whose nearest neighbors are sought.\item[{\em max\_\-dist\_\-sq}]the square of the maximum distance to the nearest neighbors.\item[{\em q}]the maximum number of points to be retured.\item[{\em dim}]the dimension of the data.\end{description}
\end{Desc}
\begin{Desc}
\item[Returns:]A pointer to a priority queue of the points found, or NULL in case of problems. \end{Desc}
\hypertarget{kdtree__cartesian_8c_23e1d72de397f86f0b13b2cc4d6c72d0}{
\index{kdtree\_\-cartesian.c@{kdtree\_\-cartesian.c}!kd\_\-range@{kd\_\-range}}
\index{kd\_\-range@{kd\_\-range}!kdtree_cartesian.c@{kdtree\_\-cartesian.c}}
\paragraph[kd\_\-range]{\setlength{\rightskip}{0pt plus 5cm}struct {\bf pqueue}$\ast$ kd\_\-range (struct {\bf kdNode} $\ast$ {\em node}, \/  float $\ast$ {\em p}, \/  float $\ast$ {\em max\_\-dist\_\-sq}, \/  int {\em dim}, \/  int {\em ordered})\hspace{0.3cm}{\tt  \mbox{[}read\mbox{]}}}\hfill}
\label{kdtree__cartesian_8c_23e1d72de397f86f0b13b2cc4d6c72d0}


Perform a range search around a point. 

\begin{Desc}
\item[Parameters:]
\begin{description}
\item[{\em node}]the root node of the tree to be searched.\item[{\em p}]the location of the point around which the search is carried out .\item[{\em max\_\-dist\_\-sq}]the square of the radius of the hypersphere.\item[{\em dim}]the dimension of the data. \item[{\em ordered}]determines whether the result list should be ordered in increasing distance (KD\_\-ORDERED) or unordered (KD\_\-UNORDERED).\end{description}
\end{Desc}
\begin{Desc}
\item[Returns:]A pointer to a priority queue containing the points found, NULL in case of problems. \end{Desc}
